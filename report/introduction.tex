\section{Introduction} % (fold)
\label{sec:introduction}

In recent years, the advances in processing power of average home computers have made it possible to simulate realistic terrains near-realtime. This paper presents a method that combines three well-known terrain generation algorithms (i.e., Perlin Noise  \cite{perlin:2002}, Simplex Noise \cite{perlin:2001}, and Diamond Square \cite{fournier:1982}) and a JavaScript 3D engine, called Three.js \cite{threeJS}, for generating natural looking terrains on the Web. With some criteria for applicability in procedurally robot designs generation, a Web-based tool for randomly generating terrains is then presented. Finally, to create more interesting and complicated terrains, we mixed the noise and fractal functions of these well-known algorithms.     

We care about procedural terrain generation for different reasons. To name a few, one aspect of our Robot Design Generator project at the Augmented Design Lab is the testing of whether our produced robot designs are valid. Consequently, performing a test-drive of those designs on numerous simulated (procedurally generated) terrains would be a great validity test. Additionally, art assets such as terrain's textures and geometry are difficult and time consuming to generate. Therefore, by using procedural terrain generation techniques, we will be able to create many such assets with subtle modifications of parameters, such height, x, y, and z axis. Lastly, since our project is a Web-based tool, we need to deal with lots of terrain variations and memory/bandwidth requirements. Therefore, we believe in the use of procedural terrain generation techniques to satisfy those requirements.    

In the algorithms described in this paper, terrain would be represented by a 3D height maps using floating point values between $0$ and $1$. Unless otherwise stated, all examples explained in this paper used the algorithms' default noise or fractal functions. All implementations of these algorithms, as well as the overall tool, were done in JavaScript.  

This paper is structured as follows: First, we briefly review the related work. Then, we describe the overall terrain generation system and write about the tool's architecture, the specific choices we made in implementing not only the Perlin Noise, Simplex Noise, and Diamond algorithms, but also the overall tool. Finally, we conclude and then talk about the future work.  

% section introduction (end)
