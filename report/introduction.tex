\section{Introduction} % (fold)
\label{sec:introduction}

In recent years, the advances in processing power of average home computers have made it possible to simulate realistic terrains near-realtime. This paper presents a method that combines three well-known terrain generation algorithms (i.e., Perlin Noise  \cite{perlin:2002}, Simplex Noise \cite{perlin:2001}, and Diamond Square \cite{fournier:1982}) and a Javascript 3D engine, called Three.js \cite{threeJS}, for generating natural looking terrains on the Web. With some criteria for applicability in robot designs generation, a Web-based tool for randomly generating terrains is then presented. Finally, to create more interesting and complicated terrains, We mixed the noise and fractal functions of these well-known algorithms.     

We care about procedural terrain generation for different reasons. To name a few, one aspect of our robot design generator project at the Augmented Design Lab is the testing of whether the produced designs are valid. Consequently, performing a drive-test of those design on numerous simulated (procedurally generated) terrains will be a great validity test for our designs. Additionally, art assets such as terrain's textures and geometry are difficult and time consuming to generate. Consequently, by using procedural terrain generation techniques, We will be able to create many such assets with subtle modifications of parameters. Lastly, since this is a Web-based tool, We need to deal with lots of terrain variations and memory/bandwidth requirements. Therefore, We believe the use of procedural terrain generation techniques to satisfy those requirements is the logical choice.    

In the algorithms described in this paper, terrain will be represented by three-dimensional height maps using floating point values between 0 and 1. Unless otherwise stated, all examples explained in this paper use the respective default noise or fractal functions of the used terrain generation algorithms. All implementations of these algorithms, as well as the overall tool, were done in Javascript.  

This paper is structured as follows: First, We briefly review the related work. Then, We describe the overall terrain generation system and write about the tool's architecture, the specific choices We made in implementing not only the Perlin Noise, Simplex Noise, and Diamond algorithms, but also the overall tool. Finally, We conclude and then talk about the future work.  

% section introduction (end)
