\subsection{Problem Description}
Our Web-based terrain generator performs terrain generation in two steps and allows users to have a full first-person experience of a realistic, interactive virtual environment in the third step. First, it randomly pick the algorithm that will set the world. Subsequently, it randomly pick a texture generation strategy, i.e., procedurally generated texture, and loading an existing texture from disk. After the world has been set, our terrain generator tool applies a few visual effects (e.g., fog) and animations on objects part of the generated world to create a sense of immersion for users interacting with our tool. Some of the navigation operations that users have at their disposal when interacting with our generated terrains include: rotating the world, moving forward, moving backwards, and moving left/right.  

Each of the three steps described above is pictured as a component (s) of the architecture diagram shown in Figure~\ref{fig:arch} and is described in detail in the following paragraphs.

% subsection problem description (end)
