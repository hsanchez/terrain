\documentclass{acm_proc_article-sp}

\usepackage{listings}
\usepackage{color}
\usepackage{subfigure}

\definecolor{lightgray}{rgb}{.9,.9,.9}
\definecolor{darkgray}{rgb}{.4,.4,.4}
\definecolor{purple}{rgb}{0.65, 0.12, 0.82}

\lstdefinelanguage{JavaScript}{
  keywords=[1]{typeof, new, catch, function, return, for, switch, var, if, in, while, do, else, case, break},
  keywordstyle=[1]\color{blue}\bfseries,
  keywords=[2]{this, document, window, null, true, false},
  keywordstyle=[2]\color{green}\bfseries,
  identifierstyle=\color{black},
  sensitive=false,
  comment=[l]{//},
  morecomment=[s]{/*}{*/},
  commentstyle=\color{purple}\ttfamily,
  stringstyle=\color{red}\ttfamily,
  morestring=[b]',
  morestring=[b]"
}

\lstset{
   language=JavaScript,
   backgroundcolor=\color{lightgray},
   extendedchars=true,
   basicstyle=\footnotesize\ttfamily,
   showstringspaces=false,
   showspaces=false,
   tabsize=2,
   breaklines=true,
   showtabs=false,
   captionpos=b,
   columns=fullflexible
}

\begin{document}
\title{A Web-based Random Terrain Generator}
\numberofauthors{2}
\author{
  \alignauthor Zhongpeng Lin \\
  \affaddr{UC Santa Cruz} \\
  \email{linzhp@soe.ucsc.edu}
  \alignauthor Huascar A. Sanchez \\
  \affaddr{UC Santa Cruz} \\
  \email{hsanchez@cs.ucsc.edu}
}

\maketitle
\begin{abstract}
The aim of this paper is to describe a system that combines three well-known terrain generation algorithms (i.e., Perlin Noise, Simplex Noise, and the Diamond Square algorithms), and a JavaScript 3D engine, called Three.js. Our Web-based terrain generator performs terrain generation in two steps. First, it randomly pick the algorithm that will set the world. Second, our terrain generator tool applies a few visual effects (e.g., fog) and animations on objects part of the generated world to create a sense of immersion for users interacting with our tool. Our tool allows a full first-experience of a realistic, interactive virtual environment by means of some basic navigation operations, such as world rotation, and world navigation (e.g., moving forward, moving backwards, moving to the left, and moving to the right). 
\end{abstract}

\category{H.4}{Information Systems Applications}{Miscellaneous}
\category{D.2.8}{Software Engineering}{Metrics}[complexity measures, performance measures]

% TODO (whoever) change to the apropropriate terms..
\terms{Algorithms, Terrain, Procedural-Content-Generation} 

\keywords{Terrain Generator, Three.js, WebGL} 

\section{Introduction} % (fold)
\label{sec:introduction}

In recent years, the advances in processing power of average home computers have made it possible to simulate realistic terrains near-realtime. This paper presents a method that combines three well-known terrain generation algorithms (i.e., Perlin Noise  \cite{perlin:2002}, Simplex Noise \cite{perlin:2001}, and Diamond Square \cite{fournier:1982}) and a JavaScript 3D engine, called Three.js \cite{threeJS}, for generating natural looking terrains on the Web. With some criteria for applicability in procedurally robot designs generation, a Web-based tool for randomly generating terrains is then presented. Finally, to create more interesting and complicated terrains, we mixed the noise and fractal functions of these well-known algorithms.     

We care about procedural terrain generation for different reasons. To name a few, one aspect of our Robot Design Generator project at the Augmented Design Lab is the testing of whether our produced robot designs are valid. Consequently, performing a test-drive of those designs on numerous simulated (procedurally generated) terrains would be a great validity test. Additionally, art assets such as terrain's textures and geometry are difficult and time consuming to generate. Therefore, by using procedural terrain generation techniques, we will be able to create many such assets with subtle modifications of parameters, such height, x, y, and z axis. Lastly, since our project is a Web-based tool, we need to deal with lots of terrain variations and memory/bandwidth requirements. Therefore, we believe in the use of procedural terrain generation techniques to satisfy those requirements.    

In the algorithms described in this paper, terrain would be represented by a 3D height maps using floating point values between $0$ and $1$. Unless otherwise stated, all examples explained in this paper used the algorithms' default noise or fractal functions. All implementations of these algorithms, as well as the overall tool, were done in JavaScript.  

This paper is structured as follows: First, we briefly review the related work. Then, we describe the overall terrain generation system and write about the tool's architecture, as well as the specific choices we made in implementing not only the Perlin Noise, Simplex Noise, and Diamond algorithms, but also the overall tool. Finally, we conclude and then talk about the future work.  

% section introduction (end)

\section{Related Work} 
\label{sec:related_work}
% TODO

% section related_work (end)

\section{System Overview} 
\label{sec:system}

\subsection{Problem Description}
In this project, we want to produce 3D terrain on web browsers. The system needs to provide a first-person experience for users to walk on or fly over different terrain styles, which are generated randomly, with different shapes and textures. The terrain generation should not require users to install additional browser plugins and the generation calculation is done in browsers. The systems should respond to users navigation operations (mouse dragging, key pressing) quickly.
% subsection problem description (end)

\subsection{WebGL and Three.js}
In order to render 3D objects on browsers without any plug-in, browser support is indispensable. The \texttt{canvas} element in HTML 5 introduces high performance dynamic scriptable shape rendering capability to browsers \cite{wiki:canvas}, and the \textit{WebGL} context of \texttt{canvas} element further exposes 3D graphic API, which provides JavaScript the access to the computation power of GPU, making the 3D animation fast without the use of plug-ins \cite{wiki:webgl}. WebGL has been supported by most major browsers including Mozilla Firefox, Google Chrome, Safari and Opera.

However, programming with shaders to access the WebGL 3D graphic API to not an easy task. Even for simple object rendering, one needs to manually set all vertex position, construct model-view matrix and projection matrix, so as to transform the object from its local coordinate space into world space using model-view matrix, and then from world space to camera space using project matrix \cite{Lengyel2012}. When textures and lighting are involved, the task becomes even more complicated. Examples of programming WebGL shaders can be found in \textit{Learning WebGL} \cite{learningWebGL}.

To ease the development process, several WebGL frameworks were developed, such as Three.js\footnote{https://github.com/mrdoob/three.js/}, PhiloGL\footnote{http://www.senchalabs.org/philogl/}, GLGE\footnote{http://www.glge.org/}, SceneJS\footnote{http://scenejs.org/} and C3DL\footnote{http://www.c3dl.org/}, to name some famous ones. Among these frameworks, Three.js has the most ``watchers" on Github, an indicator of popularity, to the date of writing. In addition, it provides a large number of examples to start with and has an active community. As the result, it is the WebGL framework of choice in this project.

Three.js provides two types of renderers to render 3D scene: \texttt{CanvasRenderer} and \texttt{WebGLRenderer}. \texttt{CanvasRenderer} uses 2D context of \texttt{canvas} to render 3D scene when WebGL context is not supported, e.g. in Internet Explorer, while \texttt{WebGLRenderer} takes the advantage of the full power of GPU to render the scene, and thus making it faster in theory. As we use WebGL in this project, we need to create an instance of \texttt{WebGLRenderer} at the very beginning:
\begin{lstlisting}
var renderer = new THREE.WebGLRenderer({antialias: true});
renderer.setSize(document.body.clientWidth, document.body.clientHeight);
\end{lstlisting}

The renderer element need to be appended to an HTML element, such as \texttt{div} or the document \texttt{body}. In this project, we use \texttt{body} as container:
\begin{lstlisting}
document.body.appendChild(renderer.domElement);
\end{lstlisting}

We can also include a bit styling to make it pretty:
\begin{lstlisting}
renderer.setClearColor( scene.fog.color, 1 );
renderer.domElement.style.position = "absolute";
renderer.domElement.style.top = MARGIN + "px";
renderer.domElement.style.left = "0px";
\end{lstlisting}

Then we can make a scene with fog and add a cube into it: 
\begin{lstlisting}
var scene = new THREE.Scene();
scene.fog = new THREE.Fog(0x050505, 2000, 4000);
scene.fog.color.setHSV(0.102, 0.9, 0.825);
var cube = new THREE.Mesh(new THREE.CubeGeometry(50, 50, 50), new THREE.MeshBasicMaterial({color: 0x000000}));
cube.position.x = 30;
cube.position.y = 40;
cube.position.z = 50;
cube.rotation.x = Math.PI / 4;
scene.add(cube);
\end{lstlisting}

We only need to specify the position and rotation of 3D objects, and add them to the scene. Three.js will performance the transformation from local space to world space automatically. Similarly, we only need to set the position and orientation of camera, the transformation from world space to camera space is left to Three.js:
\begin{lstlisting}
//PerspectiveCamera(field-of-view, viewAspectRatio, nearest, farthest);
var camera = new THREE.PerspectiveCamera(45, width / height, 1, 10000);
camera.position.z = 300;
camera.lookAt(30, 40, 50);
\end{lstlisting}

Finally, the renderer render the scene from the camera:
\begin{lstlisting}
renderer.render(scene, camera);
\end{lstlisting}

At this point, the 3D cube can be rendered on the screen. Animation is simple too, with the help of \texttt{requestAnimationFrame} function:
\begin{lstlisting}
function animate() {
	requestAnimationFrame(animate);
	render();
}
\end{lstlisting}

\texttt{requestAnimationFrame} accepts a function to call when a repaint is needed next. By passing in the \texttt{animate} function, we make it called every time the canvas is repaint, and thus \texttt{render} function can be used to set the new position or rotation of objects, including the camera.


% subsection three.js (end)

\subsection{Terrain Generator Architecture}
The architecture of our terrain generator is shown in Figure~\ref{fig:arch}. At the bottom sits the WebGL API, which is used to send shader programs to GPU. Three.js is built on WebGL API to hide details of shaders and provide a convenient way to manipulate the 3D scene. Then we developed several utility functions setup the scene and camera, as well as using the height map generated by above-mentioned algorithms to render terrains. We also develop extensions to three.js to produce richer effects. On top of them, we developed a simple API to be called from HTML page.
\begin{figure}
	\center
	\includegraphics[scale=0.45]{arch.png}
	\caption{Architecture}
	\label{fig:arch}
\end{figure}
% subsection terrain generator architecture (end)

\subsection{Terrain Generation Algorithms}
Terrain generation is essentially to generate a height for each point in 2D space. It can be seen as using a function to map \textit{x} and \textit{y} coordinates into \textit{z} coordinates: \[z=terrainGen(x, y)\]

In this project, we implemented three terrain generation algorithms, namely, Perlin Noise, Simplex Noise and Diamond Square. This section gives an overview of the three algorithms.

\subsubsection{Perlin Noise}
Perlin noise is procedurally generated noise proposed by Ken Perlin \cite{Perlin2002}. In 2D space, Perlin noise works as follows:
\begin{enumerate}
	\item Create a grid of vectors, whose \textit{x} and \textit{y} coordinates are all integers. Every point in the 2D space falls into a square in the grid, where four vertexes of the square can be decided by taking the floors and ceilings of its \textit{x} and \textit{y} coordinates. 
\end{enumerate}
\subsubsection{Simplex Noise}
\subsubsection{Diamond Square}
% subsection algorithms (end)
\subsection{Demonstration}
In this section, we will present the interface of the terrain generator, with all the features described in previous sections. 

The system has only one web page as interface. The middle of it shows the first-person view of the terrain. Below the terrain is a simple help for users to control the view. Users could move forward/backward/right/left using the arrow keys, or press 'g' to regenerate another terrain. In addition, users could toggle day or night view by pressing 'n' key. 

Figure~\ref{fig:demo_0_1} was generated using the Perlin Noise algorithm and covered by a grass texture. We could see the surface of the terrain was very smooth, with high mountains. Figure~\ref{fig:demo_1_1}, \ref{fig:demo_1_2} and \ref{fig:demo_1_3} were generated by the Diamond-square algorithm. The difference between both generated terrains was the used texture. You could also see these terrains were not as smooth as the one generated by the Perlin Noise algorithmn. These generated terrains were mostly rolling hills. 

Figure~\ref{fig:demo_2_1} was generated by combining the Perlin Noise and Diamond-square algorithms. The generated terrain combines the terrain features of the two: high mountains with rough surface. Finally, the Simplex noise algorithm was used by our generator to generate sharp mountains as shown in Figure~\ref{fig:demo_3_0}, \ref{fig:demo_3_2} and \ref{fig:demo_3_3}.

\begin{figure*}
	\center
	\includegraphics[scale=0.4]{images/demo_0_1.png}
	\caption{Perlin Noise with grass texture}
	\label{fig:demo_0_1}
\end{figure*}
\begin{figure*}
	\center
	\includegraphics[scale=0.4]{images/demo_1_1.png}
	\caption{Diamond-square with grass texture}
	\label{fig:demo_1_1}
\end{figure*}
\begin{figure*}
	\center
	\includegraphics[scale=0.4]{images/demo_1_2.png}
	\caption{Diamond-square with sand texture}
	\label{fig:demo_1_2}
\end{figure*}
\begin{figure*}
	\center
	\includegraphics[scale=0.4]{images/demo_1_3.png}
	\caption{Diamond-square with rock texture}
	\label{fig:demo_1_3}
\end{figure*}
\begin{figure*}
	\center
	\includegraphics[scale=0.4]{images/demo_2_1.png}
	\caption{Combining Perlin noise and diamond-square with grass texture}
	\label{fig:demo_2_1}
\end{figure*}
\begin{figure*}
	\center
	\includegraphics[scale=0.4]{images/demo_3_0.png}
	\caption{Simplex noise}
	\label{fig:demo_3_0}
\end{figure*}
\begin{figure*}
	\center
	\includegraphics[scale=0.4]{images/demo_3_2.png}
	\caption{Simplex noise with sand texture}
	\label{fig:demo_3_2}
\end{figure*}
\begin{figure*}
	\center
	\includegraphics[scale=0.4]{images/demo_3_3.png}
	\caption{Simplex noise with rock texture}
	\label{fig:demo_3_3}
\end{figure*}
% subsection demo (end)


% section system overview (end)


\section{Conclusions} % (fold)
\label{sec:conclusions}
% TODO

% section conclusions (end)

\section{Future Work} 
\label{sec:future_work}
% TODO

% section future work(end)


\bibliographystyle{abbrv}
\bibliography{report}

\end{document}
