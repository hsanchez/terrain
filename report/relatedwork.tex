\section{Related Work} 
\label{sec:related_work}

There has been a large amount of research in the area of terrain generation. Traditionally most of it has focused on generating terrain by using fractal techniques, like Olsen's \cite{lechner:2006}, Szeliski and Terzopoulos's \cite{szeliski:1989}, and Pi's \cite{pi:2006}.

Olsen \cite{lechner:2006} discussed the use fractal Brownian motion and perturbed Voronoi diagrams in erosion algorithms. Szeliski and Terzopoulos's \cite{szeliski:1989} used real digital elevation data which they perturb using splines to address fractal terrain generation. Pi \cite{pi:2006} created fractal landscapes using Perlin noise. As with most fractal-based approaches, their algorithms do not appear to be controllable.

Li, Wang, Zhou, Tang, Yang \cite{li:2006} used machine learning to model example heightmaps. These models were then used by the authors to come up with new interesting terrains. Unfortunately, no good deed goes unpunished. A major issue with this approach was a frequent unclearness of what needed to be learned or what features were important to be part of the training data.

Ong \cite{ong:2005} used genetic algorithms to generate terrains. Besides focusing on controllability, their approach sketched the boundary of the terrain to be generated. The main form of contrabillity of this approach was a database containing a set of representative heightmap samples, which could be controlled by the designer.
  
Lastly, Doran and Parberry \cite{doran:2010} focused on allowing the designer more natural control over the generated terrains without sacricing too much the desirable Procedural Content Generation attributes, such as novelty, structure, interest, and speed.

The aim of this paper is to provide a method that combines three well-known terrain generation algorithms (i.e., Perlin Noise  \cite{perlin:2002}, Simplex Noise \cite{perlin:2001}, and Diamond Square \cite{fournier:1982}) and a Javascript 3D engine, called Three.js \cite{threeJS}, for generating natural looking terrains on the Web.

% section related_work (end)
