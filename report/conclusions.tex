\section{Conclusions} % (fold)
\label{sec:conclusions}

This paper presented a Web-based random terrain generator tool. Our demo showed all the features described in this paper and gave the audience a chance to experience the generation of different terrains based on noise functions, such as the Perlin and Simplex noise, fractal functions, such as the Diamond Square algorithm, and the combination of noise-and-fractal functions. 

Our screenshots showed that the produced terrains were indeed realistic. Without doubt the used procedural terrain generation techniques nicely scaled in our Web-based random terrain generator when generating random terrains. 

This hybrid of flexible noise and fractal functions, making use of a powerful 3D engine, may lead to more tunable terrains that could meet certain testing criteria needed to validate robot designs produced by our Robot Design Generator tool. 

Still lacking in our tool are: useful features for quickly and cogently indicating what function(s) was used to set the shape of the world, and controllable parameters that could influence the overall elevation, roughness, and local details of the generated terrains. These features would be eventually implemented and then included in the final Robot Design Generator project. 


% section conclusions (end)
